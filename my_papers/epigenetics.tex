\documentclass[11pt,twocolumn]{article}
\title{The impact of epigenetics on who we are}
\author{Christoph Pickl}
%\date{\today}
\date{July 22, 2021}

\usepackage{hyperref}
\usepackage{enumitem}

\begin{document}
\maketitle

\subsection*{Abstract}

TODO

\textbf{Keywords}: Epigenetics, Transgenerational Inheritance, Behavioral Epigenetics, Psychology, Personality Traits, Development Biology, Determinism, Behaviorism

\section{Introduction}

This paper was written in Amsterdam by Christoph Pickl while being a psychology bachelor degree student at the Dutch \href{https://www.ou.nl/en/}{Open University}. It was conducted without any supervisor but has been reviewed by acquaintances with a formal and professional background in psychology. There are no conflicts of interest present.

\subsection{Definition}

\textit{``In biology, epigenetics is the study of heritable phenotype changes that do not involve alterations in the DNA sequence. The Greek prefix epi- in epigenetics implies features that are "on top of" or "in addition to" the traditional genetic basis for inheritance.''} -- \href{https://en.wikipedia.org/wiki/Epigenetics}{Wikipedia}.

\subsection{Motivation}

As of several occurences in the author's personal life, having pro- and con-conversations about this topic, including the possibility that our personal choices on who we are, are not decided by free will but -- at least partly -- more because of genetical inheritance determined by the behavior and experiences of our ancestors. These daily life conversations did not take any productive direction and lead to no useful and reliable conclusions.

\subsection{Goal}

This paper tries to give an overview of the current scientific findings which have been agreed upon by the scientific community to provide a more firm foundation for a more fact-driven -- rather a ``want-believing-want-to-believe-driven'' -- basis for future conversations and possible reasoning to extend one's own knowledge about epigenetics. It should be quick and easy to read to also the non-scientific community, so common assumptions can be proven/neglected, resulting in common ground to build upon. The goal is not to prove the one or the other side is right (``Is epigenetics a thing or not?'') but rather tries to investigate to which extent it influence us.

\subsection{Status Quo}

TODO % current research state, reveal gap

\subsection{Hypothesis}

It is being claimed that our personality traits are subconsciously affected by our genes (activation rather structure), which in turn can be shaped by life experiences (of our ancestors), leading to something like ``genetical plasticity''. The fact that we do and believe what we do and believe, is more shaped by biology, rather than free will, than we might want to assume. In order to extremely simplify the claim, let's take these two examples:

% or arabic / roman
\begin{enumerate}[label=\alph*)]
	\item A person living in Austria, who's grand-grand parents have been constantly in fight with Turkish people, might lead to an increased probability of prejudice about those people.
	\item A person living in the Netherlands might develop an innate tendency to trade and to conquer as of the cultural history of globalism, capitalism, colonization and imperialism.
\end{enumerate}

The dilemma of ``nature vs nurture'' is answered with a tendency towards nature, not only via culture as a medium, but genetics.

\section{Method}

TODO

\section{Results}

TODO

\section{Discussion}

TODO

\subsection{Conclusion}

TODO

\end{document}
