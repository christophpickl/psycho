\documentclass[12pt]{article}
\title{Psychology / Year 1 / Pre-Reflection}
\author{Christoph Pickl}
%\date{\today}
\date{July 3, 2021}

\newcounter{questionCounter}
\newcommand{\questionSection}[1]{\setcounter{questionCounter}{0}\section*{#1}}
\newcommand{\question}[1]{\addtocounter{questionCounter}{1}\subsubsection*{\arabic{questionCounter}. #1}}

\begin{document}
\maketitle

\subsection*{Abstract}

This document's goal is to increase my level of \textbf{awareness}, of what is going on in the inside, to get some more \textbf{clarity}.
Being aware of my primary, \textbf{personal resources} I can use during the study, and also what kind of \textbf{challenges} to focus on which can occur during the year.

Also to verify whether I am still \textbf{on-track} with what I am doing, by \textbf{comparing} the Post-Reflection with the corresponding Pre-Reflection document.
Finally, because of simple \textbf{curiosity}, to see the change that happens over the years.

\questionSection{Reflection}

\question{What is my motivation to study psychology?}

Primarily my love towards knowledge.
Because I judge it as useful information, given by nature, and as well universal, thus justified to invest time and energy into it.

I want to understand existence, life, humans, the thinking, the doing. Always pondering about this simple question: \textit{Why?}

It might help me to overcome my own personal struggles with the human nature.
My very own one I experience when being by myself, as well as social interactions with other human beings.
To understand myself and others. To be able to reason about.

Lastly, also because of ego reasons. To feel superior. To feel knowledgable.
To be heard, to be seen and respected for what I am and what I have to say. To compensate for the internal feeling of lack of competence.

\question{What/Who inspires me in my studies?}

The overlap with philosphy and spirituality.
Great thinkers like Erich Fromm, Alan Watts, Marcus Aurelius.
And as well my dear friends who also studied psychology: Marlene and Silvia.
I wonder whether it is a feeling of competition/jealousy which is (at least "also") driving my motivation?!

\question{What do I love/hate about psychology/the study?}

I love its ability to give certainty through understanding.
The relief to finally understand. And it sounds all very smart.

I hate its closed-mindedness, not being open to novel ideas.
To overcome the limitations of so-called "western thinking" (thank you Aristotle), causal thinking, distinct events, not having paradoxical logic available.
To, at least, assume the impossible, as history showed so often, we continuously fail and correct.
Yet I understand the reason why \ldots Also its focus on its own roots, not incorporating inter-cultural insights.
There is a long tradition, established over many more years than western psychology.

Studying at a university gives me the guidance I seek for. Clear structure, a clear path.
Along with deadlines, creating pressure to do my study in-time.
Somehow reminds me of the use of sprints in software engineering \ldots

Whereas the academic approach can be sometimes quite far from reality.
Also research is something I am not primarily looking for, but knowledge itself, for its own sake.
The teachings might be too broad, and not overlapping with my own, individual interests.

\question{How would I evaluate my current level of competence?}

Because of my background in philosophy, medicine, as well as my own personal curiosity and self-investigation my whole life, there is something existing at least.
Reading a few books, watching some informational videos is still by far nothing which I would consider as competence.
There is no sign of professional foundation, historical context, complete overview of the field of psychology.

\question{What are my biggest accomplishments?}

Overcoming my fear to actually register for a university.
The fear of failing, the fear of it might be too much next to work and my beloved hobbies. Simply put: courage.

My self-discipline along with a structured approach help me to self-study continuously.

\question{What are some skills/behaviors I need to improve on?}

Definitely learning to set realistic goals.
Too high expectations lead to exhaustion, demotivation, frustration and ultimately quitting the whole project.
I want to learn to readjust in case its too much. Moderation. To go slow enough to be able to keep up the marathon.

\questionSection{Pre-Reflection}


\question{What do I want to have learned after this year?}

A basic overview of what psychology is. Knowing the most famous people and studies. The main topics, ideas and schools. The subfields existing within it. And all this should be available right at my fingertip without hesitation to enumerate those.

\question{What do I look forward the most?}

Being able to claim: "I know". Although I don't really, but at least have a subjective feeling of having improved, having gained a tiny fragment more of what's out there. To basically know how little I know.

To have the feeling of success, to have something achieved, formally. To be able the set a checkmark next to a bullet point on my endlessly todo list.

\question{Where do I expect struggles and how to overcome them?}

Definitely in the collaboration with fellow students. My high expectations will very likely create some frictions. Most co-students might want to study more slowly and superficially than I do. My clear focus on the topic itself, rather socializing small talk is something some people might consider as rude, whereas my love towards knowledge is simply bigger than my love towards people.

Clearly communicating my expectations regarding focus (on topic, less personal) as well as progress (in-detail and fast) will help with a more harmonious interaction, as well as understanding from my side for them. That we are all different, with different values.

\question{How much time/energy/effort am I planning to invest?}

Most likely too much \ldots

\question{How do I measure my success?}

As this is a formal education, with a formal exam at the end, this turns out the be fairly simply answered: passing the exam. Without any cheating what so ever, making it as hard as possible -but also not unnecessarily hard. Definitely doing the exam on-site, in a well rested state of mind.

Besides that, a subjective feeling of increased competence, being able to converse about the topic in a more sophisticated way than I was before the year.

\question{Do I feel prepared for the next year? What might be missing?}

Having the study material already, started a study group for motivation and information exchange, regularly checking the resources and having my structure already in place gives me confidence in my preparation.

A buddy, a guide, a mentor, or maybe the study group could help out here, who reminds me of essential informations which I might have missed out, like resources, dates or crucial events, because of not reading all the messages and websites.

\end{document}
